\documentclass[a4paper, 12pt]{article}

\usepackage[portuges]{babel}
\usepackage[utf8]{inputenc}
\usepackage{amsmath}
\usepackage{indentfirst}
\usepackage{graphicx}
\usepackage{multicol,lipsum}

\begin{document}
%\maketitle

\begin{titlepage}
	\begin{center}
	
	%\begin{figure}[!ht]
	%\centering
	%\includegraphics[width=2cm]{c:/ufba.jpg}
	%\end{figure}

		\Huge{Universidade de São Paulo}\\
		\large{IFSC}\\ 
		\large{Lab. de Física 1}\\ 
		\vspace{15pt}
        \vspace{95pt}
        \textbf{\LARGE{Relatório 1 : Instrumentos, medidas e incertezas }}\\
		%\title{{\large{Título}}}
		\vspace{3,5cm}
	\end{center}
	
	\begin{flushleft}
		\begin{tabbing}
			Aluno: Ana Clara Amorim  Nº USP : 10691992 \\
			Aluno: Victor Silva  Nº USP : 11320901\\
			Professor : Iouri Poussep \\
	\end{tabbing}
 \end{flushleft}
	\vspace{1cm}
	
	\begin{center}
		\vspace{\fill}
			 março\\
		 2020
			\end{center}
\end{titlepage}
%%%%%%%%%%%%%%%%%%%%%%%%%%%%%%%%%%%%%%%%%%%%%%%%%%%%%%%%%%%
\newpage
\pagenumbering{arabic}
% % % % % % % % % % % % % % % % % % % % % % % % % % %
\section{Objetivo}
Esta prática experimental teve como objetivo a realização de medidas diretas, para obtenção de comprimento, massa e volume. Tal como de medidas indiretas para obter, o volume e a densidade dos materiais. Determinando através de critérios e métodos a incerteza das medidas.
\section{Materiais e Métodos}
Os instrumentos usados para aferir as medidas dessa prática, são : o paquímetro, o micrômetro e a proveta.

Para medida direta atribui-se ao objeto o valor numérico lido no instrumento.

Ja na medida indireta se faz a manipulação dos dados colhidos diretamente e atráves de cálculos podemos realizar essa medida, tendo como exemplo no nosso experimento, a medida do volume das peças.
\section{Resultados e Discussão}
\section{Conclusões}
Com esta prática, pode-se entender as incertezas intrísecas ao método do operador e aos equipamentos de medição, como o paquímetro e o micrômetro.
\end{document}
